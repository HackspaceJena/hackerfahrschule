\documentclass[compress]{beamer}
\usetheme{Warsaw}
\usecolortheme{wolverine}
\usefonttheme[onlylarge]{structurebold}
\setbeamerfont*{frametitle}{size=\normalsize,series=\bfseries}
\setbeamertemplate{navigation symbols}{}
\setbeamertemplate{footline}
{%
\begin{beamercolorbox}[wd=0.5\textwidth,ht=3ex,dp=1.5ex,leftskip=.5em,rightskip=.5em]{author
in head/foot}%
\usebeamerfont{author in head/foot}%
\hfill\insertshortauthor%
\end{beamercolorbox}%
\vspace*{-4.5ex}\hspace*{0.5\textwidth}%
\begin{beamercolorbox}[wd=0.5\textwidth,ht=3ex,dp=1.5ex,left,leftskip=.5em]{title
in head/foot}%
\usebeamerfont{title in head/foot}%
\insertshorttitle\hfill\insertframenumber/\inserttotalframenumber%
\end{beamercolorbox}%
}
\beamertemplatesolidbackgroundcolor{black!5}
\beamertemplatetransparentcovered

\usepackage[utf8]{inputenc}
\title{Das Freie Betriebsystem Linux}
\author{Frank Lanitz}
\date{\today}
\begin{document}
\begin{frame}
	\tableofcontents{}
\end{frame}
\section{Betriebsystem}
\begin{frame}
	\frametitle{Betriebssystem}
	\begin{center}
	\includegraphics[scale=0.25]{media/500px-Operating_system_placement-de.png}
	\end{center}
	\raggedleft{\tiny aus der Wikipedia}
\end{frame}
\begin{frame}
	\begin{block}{}
		\begin{itemize}
			\item Viele verschiedene Systeme 
			\item Anwendungsspezifisch optimiert
				\begin{itemize}
					\item Echtzeit
					\item Minimalistisch
					\item General
					\item Desktop/Server
					\item Mobile Geräte 
				\end{itemize}
		\end{itemize}
	\end{block}
	\begin{block}{Beispiele}
		\begin{itemize}
			\item DOS - Disc Operating System
			\item Microsoft Windows
			\item Unixoide Systeme
				\begin{itemize}
					\item GNU/Linux
					\item *BSD: Open/Free/PC/Dragefly/Net
					\item AIX
					\item MacOS X
				\end{itemize}
		\end{itemize}
	\end{block}
\end{frame}

\section{Freie Software}
\subsection{Urheberrecht, Copyright \dots}
\begin{frame}
	\frametitle{Urheberrecht und Copyright}
	\begin{block}{Allgemein}
		\begin{itemize}
			\item 
		\end{itemize}
	\end{block}
	\begin{block}{Deutschland}
		\begin{itemize}
			\item foo
		\end{itemize}
	\end{block}
	\begin{block}{USA \& Co}
		\begin{itemize}
			\item Copyright muss angemeldet werden
			\item Rechte können aufgegeben werden
		\end{itemize}
	\end{block}
	%Aber: Ich bin kein Anwalt \dots
\end{frame}

\subsection{Freie Software}
\begin{frame}
	\frametitle{Freie Software}
	\begin{itemize}
		\item Unterscheidung zwischen Freeware und freier 
			Software (free software)
		\item “Free as in ‘freedom’, not as in ‘free beer’”
			(Richard Stallman)
		\item Verschiedene Definitionen 
		\item Merkmale von Open Source
		\begin{enumerate}
			\item Quellcode liegt in lesbarer und verständlicher Form vor
			\item Quellcode darf beliebig oft kopiert, verbreitet und 
				genutzt werden
			\item Quellcode darf verändert und in der veränderten Form 
				weitergegeben werden
		\end{enumerate}
		\item Open Source Software ist kostenlos
		\item OSS ist nicht Freeware
	\end{itemize}
\end{frame}

\section{Linux}

\begin{frame}
	\frametitle{Nur der Kern}
\end{frame}

\begin{frame}
	\frametitle{Die Qual der Wahl - Distributionen wohin das Auge reicht}
\end{frame}

\begin{frame}
	\frametitle{Bezugsquellen von Linux}
	\begin{itemize}
		\item Download aus dem Internet
		\item Kauf eines Paketes bei z.B. MediaMarkt, Amazon, Thalia
		\item Vorinstalliert (z.B. IBM)
	\end{itemize}
\end{frame}

\section{Hilfe bei Problemen}
\begin{frame}
	\frametitle{Wo gibt es Hilfe}
	\begin{itemize}
		\item Handbücher, Fachbücher, Zeitschriften
		\item Internet: Newsgroups, Foren, Mailinglisten, IRC
		\item Kommerzeller Support
		\item Nette Leute, die gerne weiterhelfen (Linux User Group, Freunde \& Bekannte)
	\end{itemize}
\end{frame}



\end{document}
